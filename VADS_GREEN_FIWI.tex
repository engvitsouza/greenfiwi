\documentclass[conference,compsoc]{IEEEtran}
%\usepackage[brazilian]{babel}
%\usepackage[utf8]{inputenc}
\usepackage[T1]{fontenc}
\usepackage{lineno,hyperref}
\usepackage{amssymb}
\usepackage[inline]{enumitem}
\usepackage[linesnumbered,ruled,vlined]{algorithm2e}
\usepackage{epstopdf}
\usepackage{amsmath}
\usepackage{tabularx}
\usepackage{multirow}
\usepackage{graphicx,url}
\usepackage{filecontents}
\usepackage{cite}
\begin{document}

\title{New strategies for survivable green Fiber-Wireless networks}

\author{\IEEEauthorblockN{Vitória Alencar de Souza}
\IEEEauthorblockA{Department of electrical engineering
Federal University of Pará\\
Belém - Brazil \\
Email: vitoria.souza@itec.ufpa.br}
}

\maketitle

%%%%%%%%%%%%%%%%%%%%%%%%%%%%%%%%%%%%%%%%%%%%%%%%%%%%%%%%%%%%%%%%%%%%%%%%%%%%%%%%
%%%%%%%%%%%%%%%%%%%%%%%%%%%%%%%%%%%%%%%%%%%%%%%%%%%%%%%%%%%%%%%%%%%%%%%%%%%%%%%%
%%%%%%%%%%%%%%%%%%%%%%%%%%%%%%%%%%%%%%%%%%%%%%%%%%%%%%%%%%%%%%%%%%%%%%%%%%%%%%%%
%%%%%%%%%%%%%%%%%%%%%%%%%%%%%%%%%%%%%%%%%%%%%%%%%%%%%%%%%%%%%%%%%%%%%%%%%%%%%%%%
%%%%%%%%%%%%%%%%%%%%%%%%%%%%%%%%%%%%%%%%%%%%%%%%%%%%%%%%%%%%%%%%%%%%%%%%%%%%%%%%

\begin{abstract}
As the communications has improved the human behavior also changed and it is also creating new demands and new challenges for the next generation of broadband access.
Fiber-Wireless (FiWi) broadband networks might be used in the next generation of broadband access and the challenge is provide a FiWi standart whose could be eficcient  providing high bit rates  and  also is concerned  about  the environment's needs.
\end{abstract}

\IEEEpeerreviewmaketitle

\section{Introduction}
As concerns about climate change, rising fossil fuel prices and energy security increase, companies and governments around the world are committing great efforts to develop new tecnologies for the green strategies addressing climate chance globally and facilitating low greenhouse gas(GHG) development. Currently, the GHG emissions produced by the Information and Communication Technology (ICT) industry alone are said to be equivalent to the GHG emissions of the entire aviation industry ~\cite{yu2012green}.
Then in order to provide more efficient communications services as reduce the GHG development, Fiber-Wireless (FiWi) broadband access network could be in the future a promising ‘‘last mile’’ access technology, because it might integrates wireless and optical access technologies in terms of their respective merits, such as high capacity and stable transmission from optical access technology, and easy deployment and flexibility from wireless access technology. 

Since FiWi is expected to carry a large amount of traffic and low energy bit coast, numerous traffic flows may be interrupted by the failure of network components. Thus, survivability  joint with the reduction of the GHG in FiWi is a key issue aiming at reliable, robust  and green service ~\cite{Liu201268}.



\section{Fiber-Wireless broadband access network}

\begin{itemize}
\item Optical access network
\item Wireless access network
\item Joint Fiber-Wireless broadband access network
\end{itemize}




%%%%%%%%%%%%%%%%%%%%%%%%%%%%%%%%%%%%%%%%%%%%%%%%%%%%%%%%%%%%%%%%%%%%%%%%%%%%%%%%
%%%%%%%%%%%%%%%%%%%%%%%%%%%%%%%%%%%%%%%%%%%%%%%%%%%%%%%%%%%%%%%%%%%%%%%%%%%%%%%%
%%%%%%%%%%%%%%%%%%%%%%%%%%%%%%%%%%%%%%%%%%%%%%%%%%%%%%%%%%%%%%%%%%%%%%%%%%%%%%%%
%%%%%%%%%%%%%%%%%%%%%%%%%%%%%%%%%%%%%%%%%%%%%%%%%%%%%%%%%%%%%%%%%%%%%%%%%%%%%%%%
%%%%%%%%%%%%%%%%%%%%%%%%%%%%%%%%%%%%%%%%%%%%%%%%%%%%%%%%%%%%%%%%%%%%%%%%%%%%%%%%
\section{Survivable strategies for FiWi networks}
% what means and how?


\section{Green strategies for FiWi networks}
% what means and how?


\section{Survivable-Green strategies for FiWi networks}
% what means and how?





\section{Conclusion}
The conclusion goes here.





\bibliographystyle{./IEEEtran}
\bibliography{./telecom.bib}
\end{document}

